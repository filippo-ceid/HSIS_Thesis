\begin{acknowledgements}
    Με την περάτωση της διπλωματικής μου εργασίας και την ολοκλήρωση των μεταπτυχιακών σπουδών μου θα ήθελα να ευχαριστήσω όλους όσους με στήριξαν κατά την περίοδο φοίτησής μου και με βοήθησαν με το δικό τους ξεχωριστό τρόπο να διευρύνω τους ορίζοντες μου.
    
    Πρώτα απ' όλα αξίζουν ιδιαίτερες ευχαριστίες στον επιβλέποντα καθηγητή μου κ. Δημήτρη Νικολό, ο οποίος μου έδωσε την ευκαιρία να ασχοληθώ με το κομμάτι της ερευνάς και για πρώτη φορά να παραστώ ως ομιλητής σε επιστημονικό συνέδριο.
    
    Επίσης ευχαριστώ θερμά τον συνεπιβλέποντά μου κ. Γιώργο Κεραμίδα για όλη την πολύτιμη υποστήριξη που μου πρόσφερε από τις προπτυχιακές κιόλας σπουδές μου έως σήμερα. Φυσικά να μην παραλείψω τον υποψήφιο διδάκτορα κ. Μιχάλη Μαυρόπουλο για την βοήθειά που μου προσέφερε κατά την περίοδο της ενασχόλησής μου στην έρευνα, μέσω της άριστης συνεργασίας μας.
    
    Τέλος και πάνω απ' όλα, ευχαριστώ την οικογένειά μου που με την οικονομική και ψυχολογική υποστήριξή τους, μου πρόσφεραν τη δυνατότητα να παρακολουθήσω ένα πλούσιο πρόγραμμα σπουδών και να προετοιμαστώ καταλλήλως για την επαγγελματική ενασχόληση μου στον τομέα των νέων τεχνολογιών.
\end{acknowledgements}

%----------------------------------------------------------%
\begin{abstract}
    \vspace{-3ex}
    Η τεχνική της μείωσης της τάσης τροφοδοσίας, που χρησιμοποιείται για τη μείωση της κατανάλωσης ισχύος, αυξάνει την ευαισθησία των κυκλωμάτων στις αποκλίσεις των παραμέτρων τους από τις ονομαστικές τιμές και οδηγεί στην εκθετική αύξηση του πλήθους των δυσλειτουργικών κυψελίδων. Η παρούσα διπλωματική εργασία, επικεντρώνεται στη μελέτη της συμπεριφοράς του μηχανισμού πρόβλεψης εντολών αλλαγής της ροής του προγράμματος (εντολές διακλάδωσης),όταν τα στοιχεία που τον αποτελούν εμφανίζουν σφάλματα εξαιτίας δυσλειτουργικών κυψελίδων.
    
    Παρότι ελαττώματα στο μηχανισμό πρόβλεψης εντολών διακλάδωσης δεν εμποδίζουν την ορθή εκτέλεση των προγραμμάτων, όπως αναδεικνύεται στην παρούσα εργασία, η εμφάνιση σφαλμάτων στις κυψελίδες μνήμης του Πίνακα Πρόβλεψης Προορισμού Διακλάδωσης, ο οποίος αποτελεί τμήμα του μηχανισμού πρόβλεψης, μπορεί να έχει σημαντικές επιπτώσεις στην απόδοση και την κατανάλωση ενέργειας κατά την εκτέλεση ενός προγράμματος. Στην παρούσα εργασία, πραγματοποιείται εξονυχιστική μελέτη της λειτουργίας του Πίνακα Πρόβλεψης Προορισμού Διακλάδωσης και της επίδρασης των σφαλμάτων του στην απόδοση και την κατανάλωση ισχύος, για διαφορετικά πλήθη σφαλμάτων και παραμετροποιήσεις του συστήματος. Αντιθέτως, όπως αποδεικνύεται, σφάλματα στα στοιχεία του Πίνακα Πρόβλεψης Διακλάδωσης, τα οποία αποτελούν το υπόλοιπο τμήμα του μηχανισμού, έχουν αμελητέα επίπτωση στην απόδοση και συνεπώς στη συνολική κατανάλωση ενέργειας.
    
    Για τη μείωση των επιπτώσεων που έχει η δυσλειτουργία των κελιών μνήμης του Πίνακα Πρόβλεψης Προορισμού Διακλάδωσης, παρουσιάζεται, για πρώτη φορά, ένας μηχανισμός αποφυγής της μείωσης της απόδοσης και της αύξησης της συνολικής κατανάλωσης ισχύος. Ο προτεινόμενος μηχανισμός απαιτεί ελάχιστη αύξηση σε υλικό και σχεδόν μηδενική καθυστέρηση. Επιπλέον, συνοδεύεται από κατάλληλο αλγόριθμο ώστε να επιτυγχάνεται προσαρμοστικότητα ανάλογα με το πλήθος των σφαλμάτων. Χρησιμοποιώντας τον εξομοιωτή \gem, τα μετροπρογράμματα \spec, ένα πλήθος χαρτών σφαλμάτων, και για δύο πιθανότητες σφάλματος οι οποίες αντιστοιχούν σε δύο διαφορετικές τάσης λειτουργίας πολύ-χαμηλής κατανάλωσης, παρουσιάζεται η αποτελεσματικότητα του προτεινόμενου μηχανισμού από άποψη Εντολών ανά Κύκλο ρολογιού (\ipc) και Γινομένου Ενέργειας-Κατανάλωσης (\edp), συγκριτικά με την περίπτωση μη χρήσης του.
    
    \vspace{-2ex}
    \begin{keywords}
        Ανοχή Σφαλμάτων, Δυναμική Μεταβολή Τάσης και Συχνότητας, Πολύ-Χαμηλή Τάση Λειτουργίας, Πρόβλεψη Διακλάδωσης, Πίνακας Πρόβλεψης Προορισμού Διακλάδωσης.
    \end{keywords}
    \vspace{-4ex}
    
\end{abstract}

%----------------------------------------------------------%

\begin{abstracteng}
    \selectlanguage{english}
    
    Dynamic voltage and frequency scaling (DVFS) increases the impact of process variations on memory cells reliability resulting in an exponential ramp-up in the number of malfunctioning memory cells. Current work, investigates the behavior of branch prediction unit with faulty memory cells in its components.
    
    Although being an intrinsically fault-tolerant unit (i.e., it does not affect correctness of the system), as shown in this work, the existence of faulty memory cells in a Branch Target Buffer (BTB), which is one of the branch prediction components, can damage the performance and increase the energy consumed by the executing applications. Especially, a deep analysis of the BTB function and the effect of its faults, for several fault fault probabilities and system configurations, is performed. On the other side, as evidenced, the presence of faulty memory cells in Branch Prediction Buffer (BPB), which constitutes the rest of the branch prediction, has a negligible impact in performance, and hence in total energy consumption.
    
    To remedy the negative impact of malfunctioning BTB memory cells, for the first time, a performance recovery mechanism is introduced. The proposed mechanism has both minimal hardware overheads and practically-zero additional delays. In addition, it is accompanied by an appropriate algorithm to achieve adaptability in every case of faults. Using the simulator \gem, the \spec benchmarks, a plethora of fault maps, and for two fault probabilities that correspond to two different low power consumption modes, the effectiveness of the proposed mechanism in terms of Instructions of per Circle (\ipc) and Energy-Delay Product (\edp), compared to the case where no mechanism is used for this purpose.
    
    \begin{keywordseng}
        Fault Tolerance, Dynamic Voltage and Frequency Scaling, Ultra-Low Supply Voltage, Branch Prediction, Branch Target Buffer.
    \end{keywordseng}
    
    \selectlanguage{greek}
\end{abstracteng}
