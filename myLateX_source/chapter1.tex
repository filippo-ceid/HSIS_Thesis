\chapter{Εισαγωγή}
\label{chap1}

Η συνεχώς αυξανόμενη ανάγκη για συρρίκνωση της τεχνολογίας των σύγχρονων ολοκληρωμένων κυκλωμάτων, καθώς και η προσπάθεια μείωσης της απαιτούμενης ενέργειας με ταυτόχρονη αύξηση της απόδοσης, έχουν φέρει στην επιφάνεια το πολύ σημαντικό πρόβλημα των σφαλμάτων στα στοιχεία μνήμης. Τα σφάλματα ενός ψηφιακού κυκλώματος στην καλύτερη περίπτωση ενδέχεται να έχουν απλώς μία επίπτωση στην απόδοση του συστήματος. Στην χειρότερη περίπτωση όμως μπορούν να προκαλέσουν ακόμη και την εξαγωγή λανθασμένων αποτελεσμάτων, δίχως αυτό να γίνεται αντιληπτό άμεσα. Για το λόγο αυτό έχουν αναπτυχθεί πληθώρα τεχνικών τόσο για τη μείωση τους όσο και την ανοχή τους κατά το μέγιστο δυνατόν.

%----------------------------------------------------------%

\section{Αντικείμενο της Διπλωματικής}
\label{chap1_Object}

Στην παρούσα διπλωματική μελετούνται τα σφάλματα στη Μονάδα Δυναμικής Πρόβλεψης Διακλαδώσεων των σύγχρονων υπερβαθμωτών επεξεργαστών, και αναζητείται λύση για τα σφάλματα που εμφανίζονται στον Πίνακα Πρόβλεψης Προορισμού Διακλάδωσης. Συγκεκριμένα, όπως θα παρουσιαστεί στα ακόλουθα κεφάλαια, τα σφάλματα αυτού του πίνακα παρότι δεν επηρεάζουν το τελικό αποτέλεσμα της εκτέλεσης ενός προγράμματος, μπορούν να προκαλέσουν σε ορισμένες περιπτώσεις σημαντική αύξηση του απαιτούμενη χρόνου ολοκλήρωσης του. Η αύξηση του χρόνου συνεπάγεται τη μείωση της απόδοσης καθώς και την αύξηση της καταναλισκόμενης ενέργειας.
\par
Όπως φανερώνει η μελέτη στον Πίνακα Πρόβλεψης Προορισμού Διακλάδωσης, η πτώσης της απόδοσης εξαιτίας των σφαλμάτων οφείλεται στην ύπαρξη πλήρως ελαττωματικών συνόλων, δηλαδή συνόλων των οποίων όλα τα πλαίσια είναι ελαττωματικά, με αποτέλεσμα την αδυναμία αποθήκευσης έγκυρης πληροφορίας για ένα πλήθος εντολών διακλάδωσης. Παρότι οι περιπτώσεις τέτοιων συνόλων είναι περιορισμένες, η ύπαρξή τους μπορεί να προκαλέσει πολύ μεγάλη αύξηση του συνολικού απαιτούμενου χρόνου ενός προγράμματος, οδηγώντας με αυτό τον τρόπο σε αύξηση της καταναλισκόμενης ενέργειας. Η λύση στο πρόβλημα αυτό δίνεται από την προτεινόμενη τεχνική της λογικής μετάθεσης των πλαισίων που αποτελούν τον Πίνακα Πρόβλεψης Προορισμού Διακλάδωσης, η οποία επιτυγχάνεται με ελάχιστο κόστος.

\section{Συνεισφορά}
\label{chap1_Contribution}

\begin{enumerate}[itemsep=0.5pt]
    \item Μελετήθηκε η συμπεριφορά της Μονάδας Δυναμικής Πρόβλεψης Διακλαδώσεων κατά τη λειτουργία σε κατάσταση πολύ χαμηλής κατανάλωσης, στην οποία εμφανίζονται σφάλματα εξαιτίας των ελαττωματικών στοιχείων.
    \item Προτείνεται μία ελαχίστου κόστους τεχνική επίλυσης των επιπτώσεων των σφαλμάτων του Πίνακα Πρόβλεψης Προορισμού Διακλάδωσης.
    \item Παρουσιάζεται κατάλληλος αλγόριθμος υπολογισμού σε τρεις εκδοχές, ο οποίος έχει τη δυνατότητα να προσαρμόζεται στην κάθε περίπτωση σφαλμάτων.
    \item Αξιολογήθηκε η επιτυχία κάθε εκδοχής του αλγορίθμου στην επίτευξη του στόχου.
    \item Μελετήθηκε η βελτίωση που προσφέρει η προτεινόμενη τεχνική στην απόδοση και την κατανάλωση του συστήματος όταν λειτουργεί σε κατάσταση χαμηλής κατανάλωσης.
\end{enumerate}

Τμήμα της παρούσας μελέτης έχει εκδοθεί στην ερευνητική εργασία με τίτλο $``$\textit{\en{Recovery of Performance Degradation in Defective Branch Target Buffers}}$"$, η οποία παρουσιάστηκε στο 22\textsuperscript{ο} διεθνές συνέδριο \en{On-Line Testing and Robust System Design} \cite{filippou2016recovery}.

%----------------------------------------------------------%

\section{Οργάνωση του Τόμου}
\label{chap1_Organization}

Στο Κεφάλαιο \ref{chap2} παρουσιάζονται οι μονάδες που αποτελούν τον Υπερβαθμωτό Επεξεργαστή και γίνεται ιδιαίτερη αναφορά στη Μονάδα Δυναμικής Πρόβλεψης Διακλαδώσεων. Αναλύονται τα στοιχεία που την αποτελούν, ο τρόπος λειτουργίας της καθώς και η σημασία χρήσης της.
\par
Στο Κεφάλαιο \ref{chap3} γίνεται μία γενική περιγραφή της σύγχρονης τεχνολογίας και της κατανάλωσης ενέργειας των σύγχρονων ολοκληρωμένων. Επίσης παρουσιάζεται η εμφάνιση βλαβών στις κυψελίδες \en{SRAM} κατά την προσπάθεια μείωσης της τάσης λειτουργίας των κυκλωμάτων. Τα στοιχεία που παρουσιάζονται αποτελούν τμήματα ερευνών που έχουν δημοσιευτεί.
\par
Στο Κεφάλαιο \ref{chap4} μελετάται η συμπεριφορά της Μονάδας Δυναμικής Πρόβλεψης Διακλαδώσεων όταν το ολοκληρωμένο κύκλωμα λειτουργεί σε χαμηλό δυναμικό.
\par
Στο Κεφάλαιο \ref{chap5} γίνεται η παρουσίαση της προτεινόμενης τεχνικής ανοχής σφαλμάτων, καθώς και του απαιτούμενου αλγορίθμου ώστε η τεχνική να είναι εφαρμόσιμη σε οποιοδήποτε ολοκληρωμένο διαθέτει τεχνική ανίχνευσης λαθών.
\par
Στο Κεφάλαιο \ref{chap6} παρουσιάζονται αναλυτικά τα αποτελέσματα εξομοιώσεων της προτεινόμενης τεχνικής σε έναν γενικού σκοπού επεξεργαστή. Γίνεται ανάλυση της βελτίωσης στην απόδοση καθώς και της σημαντικής μείωσης της συνολικής καταναλισκόμενης ενέργειας.
\par
Στο Κεφάλαιο \ref{chap7} περιγράφεται αναλυτικά η διαδικασία εκτέλεσης της πειραματικής διαδικασίας. Επίσης παρέχονται αναλυτικές πληροφορίες για τα εργαλεία που χρησιμοποιήθηκαν.
\par
Στο Κεφάλαιο \ref{chap8} γίνεται ο απολογισμός της διπλωματικής εργασίας καθώς και η αναφορά ενός μικρού τμήματος έρευνας η οποία βρίσκεται σε εξέλιξη, και η οποία βασίζεται στην παρούσα διπλωματική εργασία.

%----------------------------------------------------------%
